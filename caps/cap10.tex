\chapter{Referências e Estilo ABNT}

O template TED ITA-SAC utiliza o estilo de citações e referências conforme as normas da ABNT, implementado via o pacote abntex2cite em conjunto com o BibTeX. Essa abordagem permite o uso padronizado de citações autor-data e a geração automática da lista de referências.

\section{Configuração do estilo}

A configuração está centralizada no arquivo:

settings/setabnt.tex

O arquivo de referências utilizado deve estar em formato BibTeX (.bib), preferencialmente localizado em:

refs/referencias.bib

\section{Exemplos de citações reais}

Abaixo estão exemplos práticos utilizando as chaves presentes no arquivo `.bib`:

Segundo \citeonline{murca2020characterizing}, a análise do desempenho do espaço aéreo brasileiro pode ser realizada por meio de dados de trajetória.

Estudos sobre priorização de investimentos em aeródromos da aviação geral são apresentados em \cite{caetano2022criteria}.

A ineficiência vertical em procedimentos de descida foi discutida por \cite{szenczuk2021causalvertical} sob uma abordagem causal.

\section{Citação múltipla}

Também é possível citar múltiplas referências no mesmo trecho:

Diversos autores têm contribuído para o tema da infraestrutura e operação aérea no Brasil \cite{murca2020characterizing, caetano2022criteria, szenczuk2021causalvertical}.


\section{Compilação com BibTeX}

A sequência recomendada para compilar e gerar corretamente as referências é:

1. Rodar `xelatex main.tex`  
2. Rodar `bibtex main.aux`  
3. Rodar `xelatex main.tex` duas vezes

Se estiver usando o Makefile do template, isso já está automatizado com:

make

\section{Observações}

- As chaves entre chaves `{}` devem coincidir exatamente com as do arquivo `.bib`.
- A ordem de chamada das referências será ajustada automaticamente com base na ordem de citação no texto.
- Referências não citadas no texto não aparecerão na bibliografia final.
