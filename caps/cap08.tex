\chapter{Capa e Plano de Fundo}

O template TED ITA-SAC utiliza uma capa institucional personalizada e uma marca d’água como plano de fundo nas páginas internas. Ambos os elementos foram desenvolvidos com identidade visual própria e integrados ao sistema LaTeX.

\section{Origem dos elementos gráficos}

As artes da capa e da marca d’água foram criadas na plataforma Canva. Estão disponíveis para edição e exportação no seguinte link :


A partir desse link, é possível:

- Alterar o título, logotipos ou datas na capa;
- Exportar o plano de fundo como imagem PNG ou PDF;
- Garantir a padronização gráfica do projeto TED ITA-SAC.

\section{Configuração da capa}

A capa é definida no arquivo:

settings/coverpage.tex

Exemplo de estrutura típica:

(begin titlepage)  
(begin center)  
(vspace de 4 cm)  
Título do Documento - fonte grande e negrito  
(vspace de 2 cm)  
Projeto TED ITA-SAC  
Rodapé com ITA, São José dos Campos, ano  
(end center)  
(end titlepage)


Esse arquivo pode ser ajustado para inserir novos títulos, autores, datas ou versões.

\section{Configuração do plano de fundo (marca d’água)}

O plano de fundo é definido no arquivo:

settings/background.tex

Utiliza o pacote "background" com estrutura semelhante a:

usepackage{background}  
backgroundsetup{  
	scale = 1,  
	angle = 0,  
	opacity = 0.07,  
	contents = {imagem de fundo no tamanho da página}  
}

Importante: a imagem chamada background.png deve estar presente no diretório correto.

\section{Dicas de uso e personalização}

- Para remover o fundo em páginas específicas, como a capa, use:  
backgroundsetup{contents={}}

- O fundo pode ser trocado por outra imagem com o mesmo formato;
- A opacidade pode ser ajustada com o parâmetro "opacity";
- Recomenda-se uma imagem com resolução mínima de 150 DPI.

\section{Cores na capa}

A arte da capa criada no Canva inclui uma barra inferior colorida. Essa barra deve utilizar a cor correspondente à etapa do projeto, conforme a paleta oficial (ver Capítulo 6).

Recomendação: cada entrega (relatório técnico, documento ou produto) deve seguir a cor indicada para sua etapa. Isso garante:

- Identificação clara da etapa;
- Padronização visual entre os documentos;
- Coerência com os relatórios enviados ao BNDES.

Exemplo: um relatório da Etapa 4 da Meta 1 deve utilizar a cor meta1etapa4, conforme definida no arquivo settings/setcolor.tex.
