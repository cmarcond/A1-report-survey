\chapter{Estilo de Títulos e Numeração}

A identidade visual dos títulos (capítulos, seções e subseções) é configurada por meio do arquivo:

\begin{verbatim}
	settings/settitles.tex
\end{verbatim}

Esse arquivo utiliza o pacote `titlesec` para modificar o estilo padrão do LaTeX e aplicar:

\begin{itemize}
	\item Fontes com peso e caixa específicos;
	\item Controle de espaçamentos verticais;
	\item Numeração estilizada;
	\item Alinhamento e cor.
\end{itemize}

\section{Títulos de capítulo}

Exemplo de configuração para o comando \verb|\chapter|:

\begin{verbatim}
	\titleformat{\chapter}[display]
	{\normalfont\bfseries\Huge}
	{\chaptername\ \thechapter}
	{1ex}
	{\vspace{0.5ex}}
	\titlespacing*{\chapter}{0pt}{-10pt}{20pt}
\end{verbatim}

Explicação dos parâmetros:

\begin{itemize}
	\item Usa a palavra "Capítulo" seguida do número: \verb|\chaptername \thechapter|;
	\item \verb|\Huge| aumenta o tamanho da fonte do título;
	\item Os espaçamentos antes e depois do título são definidos por \verb|\titlespacing|;
	\item A formatação usa negrito (\verb|\bfseries|) e fonte normal (\verb|\normalfont|).
\end{itemize}

\section{Títulos de seção e subseção}

Exemplo para o comando \verb|\section|:

\begin{verbatim}
	\titleformat{\section}
	{\normalfont\bfseries\Large}
	{\thesection}{1em}{}
	\titlespacing*{\section}{0pt}{2ex plus 1ex}{1ex}
\end{verbatim}

Exemplo para o comando \verb|\subsection|:

\begin{verbatim}
	\titleformat{\subsection}
	{\normalfont\itshape\large}
	{\thesubsection}{1em}{}
\end{verbatim}

Esses estilos mantêm a hierarquia visual clara e consistente entre os níveis de título.

\section{Numeração de títulos}

A numeração automática é gerada com os comandos \verb|\thesection| e \verb|\thesubsection|.

Para remover a numeração de títulos:

\begin{verbatim}
	\setcounter{secnumdepth}{0}
\end{verbatim}

Ou, para seções específicas, é possível redefinir o estilo com:

\begin{verbatim}
	\titleformat{\section}[block]
	{\normalfont\bfseries}{}{0pt}{}
\end{verbatim}

\section{Recomendações}

\begin{itemize}
	\item Use sempre espaçamentos verticais consistentes antes e depois dos títulos;
	\item Evite o uso de letras maiúsculas em todos os títulos;
	\item Alinhe o estilo dos títulos com o restante da identidade visual do projeto;
	\item Edite o arquivo `settings/settitles.tex` para ajustar todos os estilos de forma centralizada.
\end{itemize}
