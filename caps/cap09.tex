\chapter{Elementos Pré-textuais}

O template TED ITA-SAC inclui uma estrutura pré-definida para os elementos obrigatórios antes do conteúdo técnico. Essa estrutura é configurada no arquivo:

\begin{verbatim}
	settings/pretextualpages.tex
\end{verbatim}

\section{Elementos incluídos}

Por padrão, o arquivo ativa os seguintes componentes:

\begin{itemize}
	\item Sumário (comando \texttt{\textbackslash tableofcontents});
	\item Lista de figuras (comando \texttt{\textbackslash listoffigures});
	\item Lista de tabelas (comando \texttt{\textbackslash listoftables});
	\item Página de controle de versão (opcional);
	\item Marca d’água (ativada no fundo de cada página).
\end{itemize}

\section{Sumário e listas}

A geração do sumário e das listas é automática, com base nos comandos \verb|\chapter|, \verb|\section|, \verb|\caption| e similares.

Para incluir os elementos, basta inserir no arquivo principal \texttt{main.tex}:

\begin{verbatim}
	% settings/pretextualpages.tex
% Define o background exclusivo para páginas pré-textuais

\newenvironment{pretextualblock}
{
	\ClearShipoutPictureBG
	\AddToShipoutPictureBG{
		\put(0,0){
			\parbox[b][\paperheight]{\paperwidth}{%
				\includegraphics[width=\paperwidth,height=\paperheight]{capas/background_pretex.png}%
			}
		}
	}
}
{
	% settings/background.tex
% --- Imagem de fundo em todas as páginas ---

% Comando seguro para definir a imagem de fundo
\providecommand\BackgroundPic{%
	\put(0,0){%
		\parbox[b][\paperheight]{\paperwidth}{%
			\vfill
			\centering
			\includegraphics[width=\paperwidth,height=\paperheight]{capas/background.png}%
			\vfill
		}%
	}%
}

% Aplica a imagem de fundo em TODAS as páginas
\AddToShipoutPictureBG{\BackgroundPic}
 % Restaura o background padrão
}

\end{verbatim}

E no conteúdo desse arquivo, deve constar:

\begin{verbatim}
	\tableofcontents
	\listoffigures
	\listoftables
	\cleardoublepage
\end{verbatim}

\section{Folha de versão}

A folha de controle de versão pode ser incluída antes do conteúdo técnico com:

\begin{verbatim}
	\input{caps/cap_controleversao.tex}
\end{verbatim}

Esse arquivo pode conter uma tabela com histórico de alterações, como:

\begin{center}
	\begin{tabular}{|c|c|c|l|}
		\hline
		Versão & Data & Responsável & Descrição \\
		\hline
		1.0 & 01/08/2025 & Coordenação & Documento inicial \\
		1.1 & 05/08/2025 & Revisores & Correções ortográficas \\
		\hline
	\end{tabular}
\end{center}

\section{Numeração romana}

Antes do início do conteúdo principal, a numeração é configurada em algarismos romanos:

\begin{verbatim}
	\pagenumbering{roman}
	\pagestyle{empty}
\end{verbatim}

Essa numeração é trocada para arábica ao iniciar o corpo do relatório:

\begin{verbatim}
	\cleardoublepage
	\pagenumbering{arabic}
\end{verbatim}

\section{Boas práticas}

\begin{itemize}
	\item Mantenha a ordem: capa $\rightarrow$ controle de versão $\rightarrow$ sumário/listas $\rightarrow$ conteúdo;
	\item Use sempre \verb|\cleardoublepage| entre seções principais para garantir início em página ímpar;
	\item Atualize a folha de versão a cada entrega formal do projeto.
\end{itemize}
