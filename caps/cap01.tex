\chapter{Introdução ao Template TED ITA-SAC}

Este manual documenta o uso e as boas práticas associadas ao \textbf{Template ITA-SAC}, desenvolvido no contexto do projeto Estudos para a Aviação de Hoje e do Amanhã. O template foi concebido para garantir padronização, qualidade visual e conformidade com as normas técnicas e científicas exigidas em relatórios e entregas institucionais.

O modelo está implementado em \LaTeX, com modularização completa para facilitar a manutenção, customização e reuso por diferentes autores. A compilação deve ser feita utilizando o motor \texttt{XeLaTeX}, pois o sistema de fontes depende do pacote \texttt{fontspec}, utilizado para habilitar a tipografia oficial do projeto (Cheltenham ITC Pro).

\vspace{1em}

\textbf{Principais características do template:}
\begin{itemize}
    \item Estrutura modular com arquivos separados para capa, plano de fundo, layout, fontes, cores e títulos;
    \item Compatível com normas da ABNT (\texttt{abntex2cite});
    \item Fonte principal: \textbf{Cheltenham ITC Pro}, via XeLaTeX;
    \item Estilo visual limpo e profissional, com suporte a capa institucional e marca d’água;
    \item Compilação automatizada via \texttt{Makefile};
    \item Pronto para inclusão de capítulos, glossários, siglas e bibliografia.
\end{itemize}

\vspace{1em}

\textbf{Estrutura de diretórios:}

\begin{verbatim}
	.
	|- main.tex                 % Arquivo principal
	|- settings/                % Configurações gerais
	|  |- background.tex        % Marca d’água e fundo
	|  |- coverpage.tex         % Página de capa
	|  |- fonts.tex             % Fonte Cheltenham
	|  |- ...
	|- caps/                    % Capítulos do relatório
	|- Makefile                 % Automação
\end{verbatim}

Este documento é dividido em capítulos, cada um explicando um aspecto do template, para que os autores possam modificar e estender com segurança, mantendo a padronização do projeto Estudos para a Aviação de Hoje e do Amanhã.


