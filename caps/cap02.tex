\chapter{Requisitos e Instalação}

Este capítulo apresenta os requisitos técnicos para compilar e personalizar o \textbf{Template TED ITA-SAC}, bem como orientações de instalação em diferentes ambientes.

\section{Ambiente recomendado}

O template foi testado e validado em:
\begin{itemize}
    \item \textbf{Distribuição \LaTeX:} TeX Live 2023 ou superior (recomendado: TeX Live 2025);
    \item \textbf{Compilador:} XeLaTeX (\texttt{xelatex});
    \item \textbf{Editor:} TeXstudio, VSCode + LaTeX Workshop, Overleaf (parcial).
\end{itemize}

\section{Instalação no Linux (TeX Live)}

Para usuários Linux (ex: Ubuntu), recomenda-se a instalação do TeX Live completo:

\begin{verbatim}
sudo apt update
sudo apt install texlive-full
\end{verbatim}

Caso deseje controle via \texttt{tlmgr}, instale a versão mais recente diretamente do site da TeX Live:
\url{https://www.tug.org/texlive/acquire-netinstall.html}

\section{Instalação de fontes (Cheltenham ITC Pro)}

A fonte \textbf{Cheltenham ITC Pro} deve estar instalada no sistema operacional. Copie os arquivos \texttt{.otf} para o diretório de fontes do sistema (exemplo para Linux):

\begin{verbatim}
mkdir -p ~/.fonts/cheltenham
cp *.otf ~/.fonts/cheltenham/
fc-cache -fv
\end{verbatim}

No Windows, clique com o botão direito nos arquivos \texttt{.otf} e escolha \textit{"Instalar"}.

\section{Compilação com Makefile (recomendado)}

A compilação automática pode ser feita via \texttt{make}, no terminal, a partir do diretório do projeto:

\begin{verbatim}
make
\end{verbatim}

Esse comando realiza:
\begin{enumerate}
    \item Compilação com XeLaTeX;
    \item Geração de glossário (se ativado);
    \item Execução do BibTeX;
    \item Duas recompilações para acerto de referências.
\end{enumerate}

\section{Uso no Overleaf}

Para uso online:
\begin{itemize}
    \item A compilação deve ser configurada como XeLaTeX;
    \item Os arquivos de fonte \texttt{.otf} devem ser enviados ao projeto;
    \item O glossário (se usado) pode exigir configuração especial;
    \item Limitações podem ocorrer em pacotes como \texttt{background} ou \texttt{shell-escape}.
\end{itemize}

\section{Pacotes adicionais recomendados}

Certifique-se de que os seguintes pacotes estejam disponíveis:

\begin{itemize}
    \item \texttt{fontspec, xcolor, background, titlesec}
    \item \texttt{abntex2cite, glossaries-extra, biblatex, makeindex}
    \item \texttt{geometry, fancyhdr, enumitem, graphicx, hyperref}
\end{itemize}

Para instalar manualmente via \texttt{tlmgr}:

\begin{verbatim}
tlmgr install glossaries-extra abntex2cite background titlesec
\end{verbatim}


