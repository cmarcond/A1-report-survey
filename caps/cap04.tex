\chapter{Fontes e Tipografia}

A identidade visual do template TED ITA-SAC adota a tipografia Cheltenham ITC Pro como fonte principal. Essa fonte deve estar instalada no sistema operacional, pois sua ativação depende do pacote fontspec, que só funciona com o compilador XeLaTeX.

\section{Configuração no template}

A configuração da fonte é realizada no arquivo:

settings/fonts.tex

Seu conteúdo básico é:

usepackage{fontspec}
setmainfont[
Path = ./fonts/,
Extension = .otf,
UprightFont = *-Roman,
BoldFont = *-Bold,
ItalicFont = *-Italic,
BoldItalicFont = *-BoldItalic
]{Cheltenham}

Observação: a fonte precisa estar disponível no diretório "fonts/" ou instalada no sistema com esse mesmo nome (Cheltenham).

\section{Instalação da fonte}

Em sistemas Linux:
- Crie o diretório de fontes e atualize o cache:

mkdir -p ~/.fonts/cheltenham  
cp *.otf ~/.fonts/cheltenham/  
fc-cache -fv

Em sistemas Windows:
- Clique com o botão direito nos arquivos .otf e selecione "Instalar".

Em sistemas macOS:
- Clique duas vezes nos arquivos .otf e clique em "Instalar fonte".

\section{Compilador: XeLaTeX obrigatório}

A compilação deve ser feita exclusivamente com o compilador xelatex, pois os pacotes fontspec e polyglossia (ou babel com suporte a unicode) exigem esse motor para ativar fontes do sistema ou arquivos OpenType.

Para compilar no terminal:

xelatex main.tex

Ou via Makefile:

make

\section{Mensagens de erro comuns}

- Font not found: a fonte não está instalada ou o nome está incorreto.
- Missing character: There is no...: está sendo usado um caractere Unicode que a fonte atual não suporta.
- Package fontspec Error: The font "Cheltenham" cannot be found: verifique se os arquivos .otf estão acessíveis no caminho correto.

\section{Sugestões para fallback}

Se não desejar usar Cheltenham, é possível substituir por outra fonte compatível. Exemplos:

setmainfont{TeX Gyre Pagella}  
ou  
setmainfont{Times New Roman}

Importante: fontes padrão como Times ou Latin Modern não representam a identidade visual original do projeto TED ITA-SAC.
