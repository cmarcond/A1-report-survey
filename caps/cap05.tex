\chapter{Layout e Margens}

O controle da geometria da página no template TED ITA-SAC é feito por meio do arquivo:

\begin{verbatim}
	settings/setlayout.tex
\end{verbatim}

Esse arquivo configura as margens, espaçamentos, cabeçalhos e rodapés, garantindo uma apresentação institucionalmente adequada e confortável para leitura.

\section{Configuração de margens}

O pacote \texttt{geometry} é utilizado para definir as dimensões da página:

\begin{verbatim}
	\usepackage[
	a4paper,
	left=3cm,
	right=2.5cm,
	top=3cm,
	bottom=2.5cm,
	headheight=16pt,
	includeheadfoot
	]{geometry}
\end{verbatim}

Essas medidas foram adotadas com base em boas práticas de editoração técnica e podem ser ajustadas conforme a necessidade.

\section{Espaçamento entre linhas}

O espaçamento é controlado por:

\begin{verbatim}
	\linespread{1.25}
\end{verbatim}

O valor `1.25` representa um espaçamento levemente expandido (equivalente a 1,5 linhas no Word). Esse valor proporciona boa legibilidade sem comprometer a economia de páginas.

\section{Cabeçalhos e rodapés}

O pacote \texttt{fancyhdr} é utilizado para personalizar cabeçalhos e rodapés:

\begin{verbatim}
	\usepackage{fancyhdr}
	\pagestyle{fancy}
	\fancyhf{}
	\fancyhead[R]{\thepage}
	\fancyhead[L]{\slshape \nouppercase{\leftmark}}
	\renewcommand{\headrulewidth}{0.4pt}
\end{verbatim}

Com essa configuração:

\begin{itemize}
	\item O número da página aparece no canto superior direito;
	\item O título do capítulo aparece no canto superior esquerdo;
	\item A linha divisória do cabeçalho é ativada;
	\item O rodapé permanece vazio.
\end{itemize}

\section{Outros ajustes úteis}

Você pode adicionar a data ou título do projeto ao rodapé, exemplo:

\begin{verbatim}
	\fancyfoot[C]{Projeto TED ITA-SAC}
\end{verbatim}

Para capítulos específicos com layout limpo (como capa ou versão), o comando abaixo pode ser usado:

\begin{verbatim}
	\thispagestyle{empty}
\end{verbatim}

\section{Sugestão de personalização}

Para suprimir o título do capítulo nos cabeçalhos de páginas pares e manter somente o número da página, use:

\begin{verbatim}
	\fancyhead[L]{}
	\fancyhead[R]{\thepage}
\end{verbatim}

Ou, para remover tudo:

\begin{verbatim}
	\pagestyle{empty}
\end{verbatim}

