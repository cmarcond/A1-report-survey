\chapter{Paleta de Cores e Estilo Visual}

A padronização das cores é essencial para manter a identidade visual do projeto TED ITA-SAC. Este capítulo apresenta a paleta cromática adotada para representar a Coordenação Geral, as Metas e as respectivas Etapas.

\section{Visualização da paleta de cores}

A imagem a seguir apresenta a estrutura da paleta de cores, organizada por metas e etapas.

Arquivo da imagem: images/paleta\_cores.png


Legenda: Códigos de cor utilizados no projeto por meta e etapa.


\section{Definição das cores no template}

As cores são definidas no arquivo:

settings/setcolor.tex

Dentro desse arquivo, as cores são registradas com o comando definecolor. Exemplo:

definecolor{meta1etapa4}{HTML}{2b61ae}

Importante: não utilize o comando usepackage{xcolor} dentro de setcolor.tex. Esse pacote já deve estar carregado no arquivo settings/usepackage.tex.

\section{Exemplo de uso no texto}

Para aplicar uma cor ao texto, por exemplo na Etapa 1 da Meta 2:

{color{meta2etapa1} Etapa 1 da Meta 2}

Para destacar uma frase com fundo colorido:

colorbox{meta1etapa4}{parbox de largura reduzida com o texto: "Texto destacado na cor da Etapa 4 da Meta 1."}

Essa caixa deve ser compacta (recomenda-se usar largura de 40% a 50% da linha para evitar que ultrapasse a margem do documento).

\section{Cuidados no uso das cores}

- Evite usar cores com baixo contraste entre fundo e texto.
- Utilize apenas as cores registradas em settings/setcolor.tex.
- Em documentos entregues oficialmente, utilize na capa a cor institucional que corresponde à etapa do projeto, conforme orientação da Coordenação Geral.
